\documentclass{article}

\usepackage[top=1in, bottom=1.25in, left=1.25in, right=1.25in]{geometry}
\usepackage{graphicx}
\usepackage[obeyspaces]{url}

\title{Spectral Response Instructions}
\author{James M. Ball}

\begin{document}
\maketitle

\section{Mounting the sample}
\begin{enumerate}
    \item Remove the cap of the sample holder and hold it so you can see the
    notched corner in the top right.
    \item Place the sample in the aperture and push it to the notched corner
    with pixel 1 in the top left corner and so you see the metal side of the
    electrodes.
    \\
    \\
    \begin{minipage}{\linewidth}
        \centering
        \includegraphics[width=7cm]{mounting_step2.jpg}
    \end{minipage}
    \item Holding the cap flat in one hand so the sample doesn't move, place
    the holder body onto the cap ensuring a tight fit.
    \\
    \\
    \begin{minipage}{\linewidth}
        \centering
        \includegraphics[width=7cm]{mounting_step3a.jpg}
        \includegraphics[width=7cm]{mounting_step3b.jpg}
    \end{minipage}
    \item Place the sample holder onto the stage so it fits between the two
    glass spacers.
    \\
    \\
    \begin{minipage}{\linewidth}
        \centering
        \includegraphics[width=7cm]{mounting_step4.jpg}
    \end{minipage}
\end{enumerate}

\section{Software}
\subsection{Running the software}
\begin{enumerate}
    \item Open the program from the HJSGroup folder on the server,
    \path{~\Labview\Spectral Response\spectral_response_main.vi}.
    \item When you run the program you will be prompted to input a username and
    an experiment title. These are needed to create folders to save data for
    the experiment.
    \item Click OK and the software is ready to use.
\end{enumerate}

\subsection{Device and measurement details}
\begin{enumerate}
    \item Fill in the boxes for Device label, Experiment variable, and
    Variable value.
    \item Select which pixel you want to measure.
    \item Input the wavelength range, wavelength increment, number of averages
    per wavelength, and mesaurement sensitivity (you should only change this
    for accurate measuremets in the sub-bandgap region).
\end{enumerate}

\subsection{Running a measurement}
\begin{enumerate}
    \item Click the Initialise Instruments button to check that the instruments
    are communicating with the computer and to initialise settings. Wait until
    the dialog prompt tells you the status of instrument connections. 

\subsection{Advanced}

\end{document}
